\chapter{附录:命题与命题的等价}

\section*{命题}

\subsection*{什么是命题}

在初中曾经学过:
能判断真假的语句(包括式子)叫\textbf{命题}。如:
\begin{enumerate}[(1)]
    \item $12>5$;\hfill     (真命题)
    \item  3是12的约数;\hfill     (真命题)
    \item $2=5$.\hfill(假命题)
\end{enumerate}

“真命题”也可以说成是“命题成立”;“假命题”也可以说成是“命题不成立”。

\subsection*{简单命题与复合命题}
上面举出的是一些比较简单的命题。再来看下面的一些命题:
\begin{enumerate}[(1)]
\setcounter{enumi}{3}
    \item $12>5$或者$12>20$;
    \item 3是12的约数,并且3还是15的约数;
    \item 5不是方程$x^2=4$的根;
    \item 若四边形是正方形,则它的四个边长相等.
\end{enumerate}

这里,命题(4)是用连词“或者”把命题“$12>5$”和命题“$12>20$”联结成了一个新命题;命题(5)是用连词“并且”把两个命题联结成了一个新命题;命题(6)是用“不是”
来表示对命题“5是方程$x^2=4$的根”的否定而得出的新命题。

“或者”、“并且”和“不是”(简记为“或”、“且”、“非”)称为\textbf{逻辑联结词}。“若”……,“则”也是逻辑联结词。

不含逻辑联结词的命题称为\textbf{简单命题}。如(1)、(2)、(3),由简单命题和逻辑联结词构成的新命题称为复合命题,如(4)、(5)、(6)、(7)。

\subsection*{命题的标记}
数学上常用小写拉丁字母$p,q,r,s,\ldots$来表示命题。这样,复合命题就有下述四种形式:
\begin{multicols}{4}
\begin{enumerate}
    \item “$p$或$q$”;
    \item “$p$且$q$”;
    \item  “非$p$”;
    \item  “若$p$则$q$”.
\end{enumerate}    
\end{multicols}

\begin{blk}
 下列复合命题中,各属上述哪一种:   
\begin{enumerate}[(1)]
\item 24是8与6的倍数;
\item 5是方程$f(x)=0$的根或是方程$g(x)=0$的根;
\item 对顶角相等;
\item $\angle A$与$\angle B$不相等。
\end{enumerate}
\end{blk}

\subsection*{复合命题的真假}
\begin{enumerate}[(1)]
    \item “$p$或$q$”——$p,q$中至少有一个为真时,就为真。如:
\begin{itemize}
\item “圆是直线形或多边形”是假命题;
\item “3是12的约数或是16的约数”是真命题,
\item “$7\ge 5$”是真命题,
\item “$3\ge 3$是真命题。
\end{itemize}
\item “$p$且$q$”——$p,q$同时为真时,才为真。如:
\begin{itemize}
    \item 
    “零既不是正数,又不是负数”是真命题;
    \item “
    $\sqrt{2}$既是正数,又是有理数”是假命题;
    \item 
    “3既是15的约数,又是18的约数”是真命题。    
\end{itemize}
    \item “非$p$”——$p$为真时,非$p$假;$p$为假时,非$p$真。
    \item “若$p$则$q$”——由$p$能推出$q$,则称“若$p$则$q$”为真,记作$p\Rightarrow q$,否则就称“若$p$则$q$”为假.
\end{enumerate}

\begin{example}
    指出下列复合命题的真假:
\begin{enumerate}[(1)]
    \item 
“$2\le 3$”(表示$2<3$或$2=3$)\hfill(真命题);
\item 
“$2\ge 2$”(表示$2>2$或$2=2$)\hfill(真命题);
\item $5\le 4$
(表示$5<4$或$5=4$)\hfill(假命题);
\item 
“$5>3$且$5>4$”\hfill(真命题);
\item 
“24既是8的倍数,又是5的倍数”。\hfill(假命题);
\item 
“三角形不是多边形”。\hfill(假命题);
\item 
“1既不是质数又不是合数”.\hfill(真命题)。
\end{enumerate}
\end{example}

\section*{开语句}
\begin{thm}{问1}
    下列含有变量的语句为什么不能叫做命题?试述理由。
\begin{enumerate}[(1)]
\item $3x<0$;
\item $x+5=8$;
\item $(x-2)(x+6)<0$.
\end{enumerate}

\end{thm}

\begin{analyze}
这些语句的真假随变量的取值而变化,如(1),当$x<0$时是真;当$x>0$时为假。也就是说,这些语句的真假当未确定变量的取值时不能判断,所以不能叫做命题。

含有变量的语句(如方程、不等式)称为\textbf{开语句}(有的书称为条件命题).如问1所列举的三个语句。开语句通常记作$p(x),q(x),r(x),\ldots$.

对一个开语句,当给变量赋值以后,这个开语句就变成了命题。如上面的(1),当令$x=5$时,为$3\x5<0$就成了一个命题。

在全集中,能使$p(x)$为真命题的变量$x$的取值范围称为开语句$p(x)$的\textbf{真值集合},记为$\{x\mid p(x)\}$.
\end{analyze}

\begin{example}
    写出问1中开语句的真值集合:
\end{example}

\begin{solution}
\begin{enumerate}[(1)]
    \item $\{x\mid p(x)\}=\{x\mid 3x<0\}=(-\infty,0)$
    \item $\{x\mid p(x)\}=\{x\mid x+5=8\}=\{3\}$
    \item $\{x\mid p(x)\}=\{x\mid (x-2)(x+6)<0\}=(-6,2)$
\end{enumerate}
\end{solution}

\section*{命题的等价}
\subsection*{命题间的逻辑关系}
若由$p$真可以断定$q$真,则称$p$能够推出$q$,记作$p\Rightarrow q$.

若$p\Rightarrow q$,且$q\Rightarrow p$,则称$p$与$q$为\textbf{等价命题},或称命题$p$与$q$\textbf{逻辑等价},记作$p\Longleftrightarrow q$. 若$p$与$q$不是逻辑等价,可记作$p\not\Leftrightarrow q$.




如在$\triangle ABC$中,$a,b,c$是$\angle A,\angle B, \angle C$的对边,则
\begin{itemize}
    \item $\angle C=90^{\circ}\Longleftrightarrow c^2=a^2+b^2$;
    \item $\angle C=90^{\circ}\not\Leftrightarrow a^2=b^2+c^2$;
    \item $\angle A=90^{\circ}\Longleftrightarrow \angle B+\angle C=90^{\circ}$.
\end{itemize}

若开语句$p(x)$与$q(x)$具有相同的真值集合,则称$p(x)$与$q(x)$为\textbf{逻辑等价},记作$p(x)\Longleftrightarrow q (x)$.

\begin{example}
    设$p(x):\; -3<2x+3\le 5$,试写出与$p(x)$逻辑等价的形式,并使结果尽量简单。
\end{example}

\begin{solution}
\[\begin{split}
    -3<2x+3\le 5 &\Longleftrightarrow -6<2x\le 2\\
    &\Longleftrightarrow -3<x\le 1\\
    &\Longleftrightarrow x\in (-3,1],
\end{split}\]
$\therefore\quad \{x\mid -3<2x+3\le 5\}=(-3,1]$
\end{solution}


\subsection*{四种命题的形式及其逻辑关系}
在初中已学过,如果原命题用“若$p$则$q$”表示($p$叫做原命题的条件,$q$叫做原命题的结论),则四种命题的形式就是:
\begin{center}
\begin{tabular}{p{.2\textwidth}p{.3\textwidth}}
    原命题&“若$p$则$q$”;\\
逆命题&“若$q$则$p$”;\\
否命题&“若非$p$则非$q$”\\
逆否命题& “若非$q$则非$p$”
\end{tabular}
\end{center}

看一个例子.如果
\begin{center}
\begin{tabular}{p{.2\textwidth}p{.4\textwidth}p{.2\textwidth}}
  原命题是  & “若$a=0$,则$ab=0$” &(真命题)\\
逆命题是 &“若$ab=0$,则$a=0$”  &(假命题)\\
否命题是 & “若$a\ne 0$,则$ab\ne 0$” &(假命题)\\
逆否命题是 & “若$ab\ne 0$,则$a\ne 0$” &(真命题)\\
\end{tabular}
\end{center}

在初中我们还学过,两个互为逆否的命题是等价的(即它们是同真或同假)。由此,
\begin{itemize}
    \item 原命题$\Longleftrightarrow$逆否命题,
    \item 逆命题$\Longleftrightarrow$否命题。
\end{itemize}

四个命题之间的关系是:
\begin{center}
\begin{tikzpicture}[>=stealth]
\node (A)[draw, rectangle, text width=2.5cm, align=center] at (0,0) {原命题\\若$p$则$q$};
\node (B)[draw, rectangle, text width=2.5cm, align=center] at (7,0) {逆命题\\若$q$则$p$};
\node (C)[draw, rectangle, text width=2.5cm, align=center] at (0,-4) {否命题\\若非$p$则非$q$};
\node (D)[draw, rectangle, text width=2.5cm, align=center] at (7,-4) {逆否命题\\若非$p$则非$q$};
\draw[<->, very thick](A)--node[above]{互逆}(B);
\draw[<->, very thick](C)--node[below]{互逆}(D);
\draw[<->, very thick](C)--node[left]{互否}(A);
\draw[<->, very thick](B)--node[right]{互否}(D);
\draw[<->, very thick](A)--node[above, sloped, pos=.25]{互为逆否}(D);
\draw[<->, very thick](B)--node[below, sloped, pos=.25]{互为逆否}(C);


\end{tikzpicture}
\end{center}






