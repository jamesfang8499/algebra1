\chapter{出版说明}

当前,中学教学改革已经深入到课程设置和教材改革领域。
我校数学教材的改革,以发展学生的数学思维为目标,以不改变现行教学大纲规定的教学闪容为前提,试图通过对知识结构及其展开方式的统盘考虑,实现整体优化。经多年反复探索、实验,编成了这套尝试融教材与教法、学法于一体的《北京四中高中教学讲义》。

这套讲义的产生可以上溯到 1982 年。从那时起,为了发展学生智能,提高数学素养,我校部分同志就开始对高中数学教学进行以教材改革为龙头,以学法教育为重点的“整体优化实验研究”。正是在这项研究的基础上,逐步形成了这套讲义编写的特色和风格。这就是:
\begin{enumerate}
    \item 为形成学生良好的认知结构,讲义的知识结构力求脉络分明,使学生能从整体上理解教材。
    \item 为了提高学生的数学素养,本讲义把数学思想的阐述放到了重要位置。数学思想既包含对数学知识点(概念、定理、公式、法则和方法)的本质认识,也包含对问题解决的数学基本观点。它是数学中的精华,对形成和发展学生的数学能力具有特别重要的意义。为此,讲义注重展现思维过程(概念、法则被概括的过程,教学关系被抽象的过程,解题思路探索形成的过程)。在过程中认识知识点的本质,在过程中总结思维规律,在过程中揭示数学思想的指导作用。力图使学生能深刻领悟教材。
    \item “再创造,再发现”在数学学习中对培养创造维能力至关重要,为引导学生积极参与“发现”,讲义在设计上做了某些尝试。
    \item 例题和习题的选配,力求典型、适量、成龙配套。习题分为 A 组(基本题)、 B 组(提高题)和 C 组(研究题)。教师可根据学生不同的学习水平适当选用。
    \item 教材是学生学习的依据。应有利于培养自学能力,本书注重启迪学法,并在书末附有全部习题的答案或提示,以供学习时参考。
 \end{enumerate}   

这套讲义在研究、试教和成书的过程中,始终得到了北京市和西城区教育部门有关领导的关怀和帮助,得到了北京师范大学数学系钟善基教授、曹才翰教授的热情指导,清华附中的瞿宁远老师也积极参与了我们的实验研究,并对这套教材做出了贡献,在此一并致以诚挚的谢意。

在编写过程中,北京四中数学组的教师们积极参加研讨,对他们的热情支持表示感谢。

这套讲义包括六册:高中代数第一、二、三册,三角、立体几何、解析几何各一册。

编写适应素质教育的教材,对我们来说是个尝试。由于水平所限,书中不当之处在所难免,诚恳希望专家、同行和同学们提出宝贵意见。

\begin{flushright}
    北京四中教学处\\
    1996 年 1 月
\end{flushright}