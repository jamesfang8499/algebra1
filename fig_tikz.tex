\begin{landscape}



\begin{tikzpicture}[>=stealth]
\node[draw, rectangle](A) at (-5, 5)[text width=2cm, align=center]{对应类型\\“1对1”\\“多对1”\\“1对多”\\其他};
\node[draw, rectangle](B) at (0, 5)[text width=2.5cm ]{从$A$到$B$的映射:任取$x\in A$,都有唯一的$y\in B$,使$\map{f}{x}{y}$,记作$\map{f}{A}{B}$};
\node[draw, rectangle](C) at (5, 5)[text width=2cm, align=center]{从$A$到$B$的一一映射\\$\map{f}{A}{B}$};
\node[draw, rectangle](D) at (9, 5)[text width=2.5cm, align=center]{映射\\$\map{f}{A}{B}$\\的逆映射\\$\map{f^{-1}}{B}{A}$};

\draw[->, very thick](A)--node[above]{任意性}(B);
\draw[->, very thick](B)--node[above]{单射}(C);
\draw[->, very thick](C)--(D);
\draw(A)--node[below]{唯一性}(B);
\draw(B)--node[below]{满射}(C);

\node[draw, rectangle] (E1) at (0,0){函数的定义};
\node[draw, rectangle] (E2) at (9,0){反函数的定义};

\draw[->, very thick](B)--node{$A$, $B$都是非空数集时}(E1);
\draw[->, very thick](D)--(E2);
\draw[->, very thick](E1)--(E2);
\end{tikzpicture}
\end{landscape}


\newpage

\begin{tikzpicture}[>=stealth]
\node[draw, rectangle](A) at (1,3+.75)[text width=1.5cm, align=center] {复合函数}; 
\node[draw, rectangle](B) at (1,0.75)[text width=1.5cm, align=center] {函数的\\定义}; 
\node[draw, rectangle](C) at (1,-3+.75)[text width=1.5cm, align=center] {函数\\表示法}; 
\draw[->, very thick](B)--(A);
\draw[->, very thick](B)--(C);

\node (E1) at (4,6)[right]{定义域(求法:归结为解不等式组)};
\node (E2) at (4,4.5)[right]{值域$\Bigg\{$};
\node (E21) at (5.25,5)[right]{值域的定义、求法};
\node (E22) at (5.25,4.5)[right]{特殊值(最大值、最小值)的求法};
\node (E23) at (5.25,4)[right]{给函数值的范围,求自变量的取值范围};
\node (E3) at (4,2.75)[right]{增减性$\Big\{$};
\node (E31) at (5.5,3)[right]{单调函数的定义、实质及判断};
\node (E32) at (5.5,2.5)[right]{单调区间的求法};
\node (E4) at (4,1.25)[right]{奇偶性$\Big\{$};
\node (E41) at (5.5,1.5)[right]{奇(偶)函数的定义、实质及判断};
\node (E42) at (5.5,1)[right]{奇偶性的应用};

\node (E5) at (4,.25)[right]{周期性(三角中待学)};

\node (E6) at (4,-1)[right]{画法$\Big\{$};
\node (E7) at (4,-2.5)[right]{看图认性质:函数性质与图象的一致性};
\node (E61) at (5.25,-.75)[right]{先讨论函数性质};
\node (E62) at (5.25,-1.25)[right]{再用描点法画图$\Big\{$};

\node (E71) at (8.5,-1)[right]{体现性质};
\node (E72) at (8.5,-1.5)[right]{标记特殊点};

\draw[decorate, very thick, decoration={brace, amplitude=10pt}](4,.25)--node[left=10pt]{性质}(4,6);

\draw[decorate, very thick, decoration={brace, amplitude=5pt}](4,-2.5)--node[left=7pt]{图象}(4,-1);

\draw[decorate, decoration={brace, amplitude=10pt}, very thick](2.5,-1.75)--(2.5,3);


\end{tikzpicture}

























